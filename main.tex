\documentclass[12pt,a4paper]{amsart}

\usepackage[T1]{fontenc}
\usepackage[utf8]{inputenc}
\usepackage[british]{babel}
\usepackage{mathtools}
\usepackage{amsthm}
\usepackage{amssymb}
\usepackage{mathrsfs}
\usepackage{enumitem}
\usepackage{tikz-cd}
\usetikzlibrary{decorations.markings}
\usepackage{float}
\usepackage[backend=biber, style=alphabetic]{biblatex}
\setcounter{biburlnumpenalty}{7000}
\setcounter{biburllcpenalty}{7000}
\setcounter{biburlucpenalty}{8000}
\addbibresource{main.bib}
\usepackage{hyperref}
\urlstyle{same}
\usepackage[noabbrev]{cleveref}

\theoremstyle{plain}
\newtheorem{thm}{Theorem}
\newtheorem*{thm*}{Theorem}
\newtheorem{lm}[thm]{Lemma}
\newtheorem{prop}[thm]{Proposition}
\newtheorem{cor}[thm]{Corollary}
\theoremstyle{definition}
\newtheorem{defn}[thm]{Definition}
\newtheorem{exmp}[thm]{Example}
\newtheorem{xca}[thm]{Exercise}
\theoremstyle{remark}
\newtheorem{rem}[thm]{Remark}
\Crefname{thm}{Theorem}{Theorems}
\Crefname{lm}{Lemma}{Lemmas}
\Crefname{prop}{Proposition}{Propositions}
\Crefname{cor}{Corollary}{Corollaries}
\Crefname{defn}{Definition}{Definitions}
\Crefname{exmp}{Example}{Examples}
\Crefname{xca}{Exercise}{Exercises}
\Crefname{rem}{Remark}{Remarks}

\title[Talk on Sites and Topologies]{Talk on Sites and Topologies}
\author[Pedro N\'{u}\~{n}ez]{Pedro N\'{u}\~{n}ez}
\address{Pedro N\'{u}\~{n}ez \newline
\indent Albert-Ludwigs-Universit\"{a}t Freiburg, Mathematisches Institut \newline
\indent Ernst-Zermelo-Straße 1, 79104 Freiburg im Breisgau (Germany)}
\email{\normalfont\href{mailto:pedro.nunez@math.uni-freiburg.de}{pedro.nunez@math.uni-freiburg.de}}
\renewcommand*{\urladdrname}{\itshape Homepage}
\urladdr{\normalfont\href{https://home.mathematik.uni-freiburg.de/nunez/}{https://home.mathematik.uni-freiburg.de/nunez}}
\thanks{The author gratefully acknowledges support by the DFG-Graduiertenkolleg GK1821 ``Cohomological Methods in Geometry'' at the University of Freiburg.}
\date{\today}

\setcounter{tocdepth}{1}
\sloppy
\makeatletter
\hypersetup{
  pdfauthor={\authors},
  pdftitle={\@title},
  colorlinks,
  linkcolor=[rgb]{0.2,0.2,0.6},
  citecolor=[rgb]{0.2,0.6,0.2},
  urlcolor=[rgb]{0.6,0.2,0.2}}
\makeatother

\begin{document}

\maketitle

\begin{abstract}
  Notes for a talk on Sites and Topologies as part of the seminar on Topos theory and Logic organized by Luca Terenzi at the University of Freiburg during the Winter Term 2021/2022.
\end{abstract}

\tableofcontents

%\begin{center}
%  \textcolor{gray}{---parts in gray will be omitted during the talk---}
%\end{center}

The main reference for this talk is \cite{sga4}.

\section{Recollections from previous talks}

\begin{itemize}
  \item Recall the notion of presite and pretopology from previous talk, and the example that different pretopologies on the same categories may have the same associated category of sheaves.
  \item Using the Yoneda Lemma, explain how to every covering family of an object in a presite can be attached a sub-functor of the corresponding representable functor, and introduce the notion of \textit{sieve} on an object in a category.
\end{itemize}

\section{Grothendieck topologies}

\begin{itemize}
  \item Define the notion of a \textit{Grothendieck topology} on a category and of \textit{site}.
\end{itemize}

\section{Examples}

\begin{itemize}
  \item Give examples including the \textit{chaotic} topology, the \textit{discrete} topology, and the topology \textit{associated} to a given pretopology.
\end{itemize}

\section{Topologies and sheaves}

\begin{itemize}
  \item Define the topology \textit{generated} by a family of sieves.
    Dually, define the \textit{finest} topology for which all presheaves in a given family are separated/sheaves.
  \item Show that this defines an order-reversing correspondence between Grothendieck topologies and categories of sheaves.
    Deduce that a pretopology and the associated topology define the same category of sheaves.
  \item Introduce the notion of \textit{canonical} and \textit{sub-canonical} topology.
    Explain how to characterize them via the Yoneda embedding.
\end{itemize}

\printbibliography
\vfill

\end{document}
