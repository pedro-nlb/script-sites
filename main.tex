\documentclass[12pt,a4paper]{amsart}

\usepackage{libertine}
\usepackage[libertine]{newtxmath}

\usepackage[T1]{fontenc}
\usepackage[utf8]{inputenc}
\usepackage[british]{babel}
\usepackage{mathtools}
\usepackage{amsthm}
\usepackage[mathscr]{euscript}
\usepackage{enumitem}
\usepackage{tikz-cd}
\usetikzlibrary{decorations.markings}
\usepackage{float}
\usepackage[backend=biber, style=alphabetic]{biblatex}
\setcounter{biburlnumpenalty}{7000}
\setcounter{biburllcpenalty}{7000}
\setcounter{biburlucpenalty}{8000}
\addbibresource{main.bib}
\usepackage{hyperref}
\urlstyle{same}
\usepackage[noabbrev]{cleveref}

\theoremstyle{plain}
\newtheorem{thm}{Theorem}
\newtheorem*{thm*}{Theorem}
\newtheorem{lm}[thm]{Lemma}
\newtheorem{prop}[thm]{Proposition}
\newtheorem{cor}[thm]{Corollary}
\theoremstyle{definition}
\newtheorem{defn}[thm]{Definition}
\newtheorem{exmp}[thm]{Example}
\newtheorem{xca}[thm]{Exercise}
\theoremstyle{remark}
\newtheorem{rem}[thm]{Remark}
\newtheorem{cav}{Caveat}
\Crefname{thm}{Theorem}{Theorems}
\Crefname{lm}{Lemma}{Lemmas}
\Crefname{prop}{Proposition}{Propositions}
\Crefname{cor}{Corollary}{Corollaries}
\Crefname{defn}{Definition}{Definitions}
\Crefname{exmp}{Example}{Examples}
\Crefname{xca}{Exercise}{Exercises}
\Crefname{rem}{Remark}{Remarks}

\title[Talk on Sites and Topologies]{Talk on Sites and Topologies}
\author[Pedro N\'{u}\~{n}ez]{Pedro N\'{u}\~{n}ez}
\address{Pedro N\'{u}\~{n}ez \newline
\indent Albert-Ludwigs-Universit\"{a}t Freiburg, Mathematisches Institut \newline
\indent Ernst-Zermelo-Straße 1, 79104 Freiburg im Breisgau (Germany)}
\email{\normalfont\href{mailto:pedro.nunez@math.uni-freiburg.de}{pedro.nunez@math.uni-freiburg.de}}
\renewcommand*{\urladdrname}{\itshape Homepage}
\urladdr{\normalfont\href{https://home.mathematik.uni-freiburg.de/nunez/}{https://home.mathematik.uni-freiburg.de/nunez}}
\thanks{The author gratefully acknowledges support by the DFG-Graduiertenkolleg GK1821 ``Cohomological Methods in Geometry'' at the University of Freiburg.}
\date{\today}

\sloppy
\makeatletter
\hypersetup{
  pdfauthor={\authors},
  pdftitle={\@title},
  colorlinks,
  linkcolor=[rgb]{0.2,0.2,0.6},
  citecolor=[rgb]{0.2,0.6,0.2},
  urlcolor=[rgb]{0.6,0.2,0.2}}
\makeatother

\begin{document}

\maketitle

\begin{abstract}
  Notes for a talk on Sites and Topologies as part of the seminar on Topos theory and Logic organized by Luca Terenzi at the University of Freiburg during the Winter Term 2021/2022.
\end{abstract}

\tableofcontents

%\begin{center}
%  \textcolor{gray}{---parts in gray will be omitted during the talk---}
%\end{center}

The main references for this talk are \cite{sga4} and \cite{stacks-project}.

\setcounter{section}{-1}

\section{Notation and conventions}

We use the font $\mathbf{C}$, $\mathbf{D}$, ...~for categories and the font $\mathscr{F}$, $\mathscr{G}$, ...~for presheaves.
We will use the font $F$, $G$, ...~for functors which are not presheaves or which we are not interested in regarding as such.
We will use greek letters for natural transformations.
We will ignore most of the set theoretic issues.

We will sometimes depict categories as follows:~objects are bullet points, morphisms are arrows and identities are not depicted at all.
So for example
\begin{center}
  \begin{tikzcd}
    & \bullet \arrow{d} \\
    \bullet \arrow{r} & \bullet
  \end{tikzcd}
\end{center}
represents a category with three objects, three identity morphisms and two morphisms from two of the objects to the third object.
This determines the category up to isomorphism of categories, so we will sometimes talk about \textit{the} category represented by such a picture.

\begin{cav}
  Such a picture does not need to be a commutative diagram.
  More generally, a diagram will only be assumed to be commutative if we explicitly say so.
\end{cav}

Here is a list with some categories that we might use:
\begin{itemize}
  \item The category $\mathbf{Set}$ of sets.
  \item The category $\mathbf{Set}_{*}$ of pointed sets.
  \item The category $\mathbf{Cat}$ of categories and functors.
  \item The category $\mathbf{Grp}$ of groups.
  \item The category $\mathbf{Ab}$ of abelian groups.
  \item The category $\mathbf{Ring}$ of commutative rings with one and ring homomorphisms sending one to one.
  \item If $R$ is a commutative ring with one, then we consider the categories $R$-$\mathbf{Mod}$ and $R$-$\mathbf{mod}$ of $R$-modules and finitely generated $R$-modules respectively.
  \item The category $\mathbf{Top}$ of topological spaces.
  \item The category $\mathbf{Top}_{*}$ of pointed topological spaces.
  \item The category $\mathbf{Sch}$ of schemes and its full subcategory $\mathbf{Aff}$ of affine schemes.
  \item If $\mathbf{C}$ is a category and $X$ is an object in $\mathbf{C}$, then $\mathbf{C}_{X}$ is the category of objects in $\mathbf{C}$ over $X$, whose objects are morhpisms $Y \to X$ in $\mathbf{C}$ and whose morphisms are morphisms $Y \to Y'$ in $\mathbf{C}$ making the corresponding triangle commute.
  \item If $\mathbf{C}$ and $\mathbf{D}$ are two categories, then $\mathbf{Fun}(\mathbf{C},\mathbf{D})$ is the category of functors between them, with morphisms given by natural transformations.
  \item If $\mathbf{C}$ is a category, then $\mathbf{C}^{\mathrm{op}}$ denotes its opposite category.
  \item If $\mathbf{C}$ is a category, then $\mathbf{PSh}(\mathbf{C}):= \mathbf{Fun}(\mathbf{C}^{\mathrm{op}},\mathbf{Set})$ is the category of presheaves on $\mathbf{C}$.
  \item If $X$ is a topological space, then $\mathbf{Op}(X)$ denotes the category corresponding to the partially ordered set of open subsets of $X$ and $\mathbf{PSh}(X) := \mathbf{PSh}(\mathbf{Op}(X))$ denotes the corresponding category of presheaves.
\end{itemize}

\section{Recollections from previous talks}

{\color[rgb]{0.2,0.2,0.6}
\begin{itemize}
  \item Recall limits and fiber products.
  \item Recall the notion of presite and pretopology from previous talk, and the example that different pretopologies on the same categories may have the same associated category of sheaves.
  \item Using the Yoneda Lemma, explain how to every covering family of an object in a presite can be attached a sub-functor of the corresponding representable functor, and introduce the notion of \textit{sieve} on an object in a category.
\end{itemize}
}

\subsection{Limits}

The limit construction takes as input a bunch of objects and morphisms between them in some category $\mathbf{C}$ and gives as an output a new object together with compatible morphisms to each of the previous objects having a nice universal property, which we can think of as this object being the largest possible object with such properties.
We need to make a few things precise here:

\begin{itemize}
  \item Let us start by making the sentence ``a bunch of objects and morphisms between them in some category $\mathbf{C}$'' precise.
    We want to consider something like
    \begin{center}
      \begin{tikzcd}
        & B \arrow{d} \\
        C \arrow{r} & A
      \end{tikzcd}
    \end{center}
    or maybe something without any morphisms like
    \begin{center}
      \begin{tikzcd}
        A & B
      \end{tikzcd}
    \end{center}
    or maybe even something without anything at all.
    We can make this notion precise using set theoretic language, e.g., a set of objects in $\mathbf{C}$ and a set of objects and a set of morphisms between them.
    But we want to make this notion precise using categorical language instead.
    A neat way to do this is by considering such a bunch of objects and morphisms in $\mathbf{C}$ as the image of a functor from some other category.
    This leads to the notion of a \textit{diagram} in $\mathbf{C}$, which is simply a functor $F \colon \mathbf{I} \to \mathbf{C}$.
    The previous three examples would correspond to taking $\mathbf{I}$ as the category
    \begin{center}
      \begin{tikzcd}
        & \bullet \arrow{d} \\
        \bullet \arrow{r} & \bullet
      \end{tikzcd}
    \end{center}
    or the category $(\bullet \quad \bullet)$ or the empty category respectively.

    So the input of the limit construction is a diagram, i.e., a functor.
  \item Let us discuss the sentence ``a new object together with compatible morphisms to each of the previous objects'' now.
    Given a diagram $F \colon \mathbf{I} \to \mathbf{C}$ in $\mathbf{C}$, we want its limit to be an object in $\mathbf{C}$ together with morphisms to  each object in the diagram making this new diagram commute when starting from this object.
    Again, we could make this precise using set theory, but it is more natural to continue using categorical language instead.
    A way to phrase this in categorical terms is using the notion of a \textit{cone} over the diagram $F$.
    We first define the auxiliary category $C(\mathbf{I})$ as the category obtained from adding a new object to $\mathbf{I}$ (together with the corresponding identity morphism) and adding a single morphism from this new object to every object in the old category $\mathbf{I}$.
    We can compose these new morphisms with the old ones, but since we have only added a single morphism from the new object to each old object, every composition starting from the new object and finishing in some other object has to agree and be equal to the single morphism that we added.
    A \textit{cone} over $F$ is then a diagram $C \colon C(\mathbf{I}) \to \mathbf{C}$ that restricts to $F$ over the full subcategory $\mathbf{I} \subseteq C(\mathbf{I})$.
    Explicitly, such a cone consists of a new diagram obtained from the old one by adding a new object $V$ and a new morphism from $V$ to every object in the old diagram in such a way that this new diagram commutes when starting from $V$.
    We can picture this by putting the old diagram in a horizontal plane (this would be the base of the cone) and placing $V$ above it (this would be the vertex of the cone):
    \begin{center}
      \begin{tikzcd}
        & &[-10mm] V \arrow{dl} \arrow[bend right=30]{ddll} \arrow{ddr} \arrow[bend left=20]{drr} & &[-10mm] \\
        & A \arrow{dl} \arrow{rrr} & & & B \\
        C \arrow[shift left=1, bend left=5]{rrr} \arrow[shift right=1, bend right=5]{rrr} & & & D &
      \end{tikzcd}
    \end{center}
    We stress again that this diagram is only supposed to commute when starting from $V$; for example, the two arrows $C \to D$ need not be equal.
    
    So the output of the limit construction is a cone over the input diagram, i.e., a new diagram which we can picture as a cone having the old diagram as a base.
  \item Finally, let us make the sentence ``the largest possible object with such properties'' precise.
    This is arguably the most imprecise sentence in our initial description and yet the most important part of the concept of limit, so it would be a good time to say last but not least.
    Cones over $F$ form a category, namely, the subcategory of the category of functors $\mathbf{Fun}(C(\mathbf{I}),\mathbf{C})$ whose objects are functors that restrict to $F$ over $\mathbf{I} \subseteq C(\mathbf{I})$ and whose morphisms are the natural transformations that restrict to the identity on $F$, meaning that the component at each object of $\mathbf{I} \subseteq C(\mathbf{I})$ is the identity on the image of the corresponding object under $F$.
    We can picture this by drawing the common base of the two cones as a single base, as shown in the following picture:
    \begin{center}
      \begin{tikzcd}
        V \arrow[bend left=10]{rrrrrr} \arrow{ddr} \arrow{drr} \arrow[bend right=15]{ddrrrr} \arrow{drrrrr} & & &[-10mm] & &[-10mm] & V' \arrow{llllldd} \arrow{lllld} \arrow{dl} \arrow[bend left=30]{ddll} \\
         & & A \arrow[shift right=1, bend right=20]{dl} \arrow[shift right=2, bend right=15]{rrr} & & & B & \\
         & C \arrow[shift right=1, bend right=20]{rrr} \arrow[shift right=3, bend right=30]{rrr} & & & D & &
      \end{tikzcd}
    \end{center}
    Again, this diagram is only supposed to commute starting from $V$ and starting from $V'$.
    
    So we have a category $\mathbf{Cones}(F)$ of cones over our diagram $F$.
    The limit of $F$, if it exists, is precisely a terminal cone, i.e., a terminal object in the category $\mathbf{Cones}(F)$.
    The reason to have used the word ``largest'' in our informal description is two-fold.
    On the one hand, we can think of a partially ordered set as a category having the same set of objects, in which there is a morphism $x \to y$ if and only if $x \leq y$.
    If this poset has a largest element, then this element is a terminal object in the corresponding category.
    More generally, the categorical product (which is a particular case of limit) of a subset of this poset, if it exists, corresponds to the infimum of this subset, which is the largest lower bound of all the elements in the subset.
    On the other hand, one can intuitively think of limits as subobjects and colimits as quotient object, because this is indeed a possible way to construct limits and colimits in many familiar categories, e.g., in the category of sets, in which the limit can be constructed as a suitable subset of the product of all sets in the diagram.
\end{itemize}

\begin{rem}
  Let $\mathbf{C}$ be a category in which all limits exist, meaning that for all $\mathbf{I}$ and all $F \colon \mathbf{I} \to \mathbf{C}$ the limit $\lim(F)$ exists.
  Then, for every category $\mathbf{I}$, we have a functor
  \[ \lim_{\mathbf{I}} \colon \mathbf{Fun}(\mathbf{I}, \mathbf{C}) \to \mathbf{C}. \]
  This functor is right adjoint to the functor
  \[ \operatorname{const}_{(-)} \colon \mathbf{C} \to \mathbf{Fun}(\mathbf{I}, \mathbf{C}) \]
  sending an object $X$ in $\mathbf{C}$ to the functor sending every object in $\mathbf{I}$ to $X$ and every morphism in $\mathbf{I}$ to the identity on $X$.
  In particular, the functor $\operatorname{lim}_{\mathbf{I}}$ preserves limits.
\end{rem}

\begin{exmp}[Terminal objects]
  Let $\mathbf{C}$ a category and let $\mathbf{I} = \varnothing$ be the empty category and $\varnothing \colon \varnothing \to \mathbf{C}$ the empty functor.
  Then $\lim(\varnothing)$ is a terminal object in $\mathbf{C}$.
  Examples of terminal objects include:
  \begin{enumerate}
    \item A singleton $\{ * \}$ in the categories $\mathbf{Set}$, $\mathbf{Set}_{*}$, $\mathbf{Cat}$, $\mathbf{Top}$, $\mathbf{Top}_{*}$, $\mathbf{Grp}$, $\mathbf{Ab}$, $\mathbf{Ring}$, $R$-$\mathbf{Mod}$, $R$-$\mathbf{mod}$, ... (with the corresponding structure on $\{ * \}$ omitted in each case).
    \item $\operatorname{Spec}(\mathbb{Z})$ in $\mathbf{Sch}$ and in $\mathbf{Aff}$.
    \item $X$ in $\mathbf{C}_{X}$.
    \item $X$ in $\mathbf{Op}(X)$.
    \item $\varnothing$ in $\mathbf{Set}^{\mathrm{op}}$.
    \item The constant presheaf with value $\{ * \}$ in $\mathbf{PSh}(\mathbf{C})$.
  \end{enumerate}
  Some categories in which terminal objects to do not exist include:
  \begin{enumerate}
    \item The subcategory of $\mathbf{Set}$ consisting of infinite sets.
    \item The category $\mathbf{Aff}_{S}$ of affine schemes over a non-affine scheme $S$.
    \item Any set with more than one element regarded as a category with only the identities as morphisms.
    \item The partially ordered set of finite subsets of an infinite set.
    \item The totally ordered set $(\mathbb{R},\leq)$.
  \end{enumerate}
\end{exmp}

\begin{exmp}[Binary products]
  Let $\mathbf{C}$ be a category and let $\mathbf{I}$ be the category $(\bullet \quad \bullet)$.
  We consider a diagram $F \colon \mathbf{I} \to \mathbf{C}$, which then consists of two objects $A$ and $B$ in $\mathbf{C}$ without any morphisms between them.
  Then $\lim(F)$ is a product $A \times B$ in $\mathbf{C}$, that is, an ojbect $A \times B$ in $\mathbf{C}$ with morphisms $A \times B \to A$ and $A \times B \to B$ such that for every other object $T$ in $\mathbf{C}$ admitting morphisms to $A$ and to $B$ in $\mathbf{C}$ there exists a unique morphism $T \to A \times B$ making the corresponding diagram commute.
  In a picture:
  \begin{center}
    \begin{tikzcd}
      & T \arrow[dashed]{d}{\exists !} \arrow[bend right=30]{ddl} \arrow[bend left=30]{ddr} & \\
      & A \times B \arrow{dl} \arrow{dr} & \\
      A & & B
    \end{tikzcd}
  \end{center}
  Examples of binary products include:
  \begin{enumerate}
    \item The cartesian product $A \times B$ (with the corresponding projections) in $\mathbf{Set}$, $\mathbf{Set}_{*}$, $\mathbf{Cat}$, $\mathbf{Top}$, $\mathbf{Top}_{*}$, $\mathbf{Grp}$, $\mathbf{Ab}$, $\mathbf{Ring}$, $R$-$\mathbf{Mod}$, $R$-$\mathbf{mod}$, ... (again omitting the induced structure on $A \times B$ in each case).
    \item The intersection $A \cap B$ in $\mathbf{Op}(X)$.
    \item The infimum $\operatorname{inf}\{ A, B\}$ in a partially ordered set (if it exists).
    \item The disjoint union $A \sqcup B$ in $\mathbf{Set}^{\mathrm{op}}$.
    \item The point-wise product
      \[ U \mapsto \mathscr{F}(U) \times \mathscr{G}(U) \]
      in $\mathbf{PSh}(\mathbf{C})$.
    \item The tensor product $A \otimes B$ (over $\mathbb{Z}$) in $\mathbf{Ring}^{\mathrm{op}}$.
    \item The spectrum of the tensor product of coordinate rings in $\mathbf{Aff}$, i.e.,
      \[ \operatorname{Spec}(A) \times \operatorname{Spec}(B) = \operatorname{Spec}(A \otimes B). \]
    \item The scheme $X \times Y$ obtained from $X$ and $Y$ by gluing the product of affine open subschemes.
  \end{enumerate}
  Some categories in which binary products do not always exist include:
  \begin{enumerate}
    \item The binary product of two sets with $5$ elements in the category of sets with at most $5$ elements does not exist.
    \item Any set with more than one element regarded as a category with only the identities as morhpisms does not have any binary products except the product of each object with itself.
  \end{enumerate}
\end{exmp}

\begin{exmp}[Products]
  Let $\mathbf{C}$ be a category and let $I$ now be a set regarded as a category with only the identities as morphisms.
  The limit of a diagram $F \colon I \to \mathbf{C}$ is then a product indexed by $I$; this time we don't spell out the universal property and jump straight to the examples.
  Examples of products include:
  \begin{enumerate}
    \item The cartesian product $\prod_{i \in I} A_{i}$ in $\mathbf{Set}$, $\mathbf{Set}_{*}$, $\mathbf{Cat}$, $\mathbf{Top}$, $\mathbf{Top}_{*}$, $\mathbf{Grp}$, $\mathbf{Ab}$, $\mathbf{Ring}$, $R$-$\mathbf{Mod}$, ... (again, we don't make precise the induced structure in each case).
    \item The point-wise product
      \[ U \mapsto \prod_{i \in I} \mathscr{F}_{i}(U) \]
      in $\mathbf{PSh}(\mathbf{C})$.
    \item The infimum $\operatorname{inf}_{i \in I} \{ a_{i} \}$ of a bounded-below collection of numbers in $(\mathbb{R}, \leq)$.
  \end{enumerate}
  Some examples of situations in which products do not exist include:
  \begin{enumerate}
    \item The product of infinitely many sets each containing more than one element does not exist in the category of finite sets.
    \item The product of infinitely many positive dimensional $k$-vector spaces does not exist in $k$-$\mathbf{mod}$.
    \item The product of the diagram $\mathbb{N} \to \mathbf{Op}(\mathbb{R}), n \mapsto (-1/n,1/n)$ does not exist in $\mathbf{Op}(\mathbb{R})$.
    \item Let $(a_{n})_{n \in \mathbb{N}}$ be a decreasing sequence of rational numbers converging to $\pi \in \mathbb{R}$.
      Then the product $\operatorname{inf}_{n \in \mathbb{N}} \{ a_{n} \}$ does not exist in $(\mathbb{Q}, \leq)$.
  \end{enumerate}
\end{exmp}

\begin{exmp}[Equalizers]
  Let $\mathbf{C}$ be a category and let $\mathbf{I}$ be the category $(\bullet \rightrightarrows \bullet)$.
  A diagram $F \colon \mathbf{I} \to \mathbf{C}$ corresponds to the data of two parallel morphisms $f \colon A \to B$ and $g \colon A \to B$, and its limit is called the equalizer of $f$ and $g$, denoted $\operatorname{Eq}(f,g)$.
  The universal property of the equalizer can be pictured as follows:
  \begin{center}
    \begin{tikzcd}
      & T \arrow[swap, dashed, bend right=30]{dl}{\exists !} \arrow{d} \arrow[bend left=30]{dr} & & \\
      \operatorname{Eq}(f,g) \arrow{r} & A \arrow[shift left=1]{r}{f} \arrow[swap, shift right=1]{r}{g} & B
    \end{tikzcd}
  \end{center}
  Examples of equalizers include:
  \begin{enumerate}
    \item The subset $\{ a \in A \mid f(a) = g(b) \}$ (with the corresponding inclusion) in $\mathbf{Set}$, $\mathbf{Set}_{*}$, $\mathbf{Cat}$, $\mathbf{Top}$, $\mathbf{Top}_{*}$, $\mathbf{Grp}$, $\mathbf{Ab}$, $\mathbf{Ring}$, $R$-$\mathbf{Mod}$, $R$-$\mathbf{mod}$, ... (as usual we don't make precise the induced structure in each case).
    \item We have $\operatorname{Eq}(0,f) = \operatorname{Ker}(f)$ in $\mathbf{Grp}$, $\mathbf{Ab}$, $R$-$\mathbf{Mod}$ and $R$-$\mathbf{mod}$.
    \item The point-wise equalizer
      \[ U \mapsto \{ s \in \mathscr{F}(U) \mid f_{U}(s) = g_{U}(s) \} \]
      in $\mathbf{PSh}(\mathbf{C})$.
  \end{enumerate}
  Some examples in which equalizers do not exist include:
  \begin{enumerate} 
    \item The equalizer of $(\bullet \rightrightarrows \bullet)$ does not exist in $\mathbf{I}$ itself.
    \item The equalizer of two morphisms $f, g \colon X \to Y$ which agree only on a finite subset of $X$ does not exist in the category of infinite sets.
  \end{enumerate}
\end{exmp}

\begin{xca}
  Let $\mathbf{C}$ be a category and let $f, g \colon A \rightrightarrows B$ be parallel arrows in $\mathbf{C}$ such that $\operatorname{Eq}(f,g)$ exists in $\mathbf{C}$.
  Show that the morphism $\operatorname{Eq}(f,g) \to A$ is a monomorhpism in $\mathbf{C}$.
\end{xca}

\subsection{Fiber products}

These are particular cases of limits, but they will be so relevant later on that they deserve their own subsection.
We take $\mathbf{I}$ to be the category
\begin{center}
  \begin{tikzcd}
    & \bullet \arrow{d} \\
    \bullet \arrow{r} & \bullet
  \end{tikzcd}
\end{center}

The limit of a diagram $F \colon \mathbf{I} \to \mathbf{C}$ is called a \textit{fiber product} in $\mathbf{C}$.
Before looking at particular examples, let us introduce some notation and terminology around fiber products.
Consider two morphisms $f \colon B \to A$ and $g \colon C \to A$ in $\mathbf{C}$.
Their fiber product is denoted by $B \times_{A} C$.
If we need to be precise about the morphisms $f$ and $g$, we can also use the notation $B \times_{f,A,g} C$.
By definition of the limit, the fiber product $B \times_{A} C$ comes with two projections $p_{1} \colon B \times_{A} C \to B$ and $p_{2} \colon B \times_{A} C \to C$ which fit into a commutative diagram as follows:
\begin{center}
  \begin{tikzcd}
    B \times_{A} C \arrow[swap]{d}{p_{1}} \arrow{r}{p_{2}} & C \arrow{d}{g} \\
    B \arrow[swap]{r}{f} & A
  \end{tikzcd}
\end{center}
Such a diagram is sometimes called a \textit{cartesian square}, and it satisfies a universal property which can be pictured as follows:
\begin{center}
  \begin{tikzcd}
    T \arrow[bend left=30]{drr} \arrow[bend right=30]{ddr} \arrow[dashed]{dr}{\exists !} & & \\
    & B \times_{A} C \arrow[swap]{d}{p_{1}} \arrow{r}{p_{2}} & C \arrow{d}{g} \\
    & B \arrow[swap]{r}{f} & A
  \end{tikzcd}
\end{center}
Spelled out, this means that if the outer (deformed) square commutes, then there exists a unique arrow (the dashed arrow) making the whole diagram commute.

\begin{rem}\label{rem:cartesian}
  If we say that the diagram
  \begin{center}
    \begin{tikzcd}
      D \arrow[swap]{d}{q_{1}} \arrow{r}{q_{2}} & C \arrow{d}{g} \\
      B \arrow[swap]{r}{f} & A
    \end{tikzcd}
  \end{center}
  is cartesian, we mean that it commutes and it has the previously discussed universal property, which in turn implies that there exists an isomorhpism $h\colon D \to B \times_{A} C$ such that $q_{1} = p_{1} \circ h$ and $q_{2} = p_{2} \circ h$.
  Limits are only unique up to isomorphism anyway, so we are running around a bit in circles here; but in practice we usually have an explicit construction of $B \times_{A} C$ in mind, and such a $D$ may not be explicitly given like that.
\end{rem}

Cartesian diagrams are sometimes marked as
\begin{center}
  \begin{tikzcd}
    D \arrow[phantom, "\square"]{dr} \arrow[swap]{d}{q_{1}} \arrow{r}{q_{2}} & C \arrow{d}{g} \\
    B \arrow[swap]{r}{f} & A
  \end{tikzcd}
\end{center}
or sometimes also as
\begin{center}
  \begin{tikzcd}
    D \arrow[phantom, "\lrcorner"]{dr} \arrow[swap]{d}{q_{1}} \arrow{r}{q_{2}} & C \arrow{d}{g} \\
    B \arrow[swap]{r}{f} & A
  \end{tikzcd}
\end{center}
A mnemonic to remember the orientation of the symbol in the middle is that the two sides of the right angle are pointing to the two morphisms which were given by the universal property of the limit.
This notation avoids confusion with the dual notion of a \textit{pushout square}, in which the two arrows created by the universal property of the colimit would be the ones pointing towards the bottom right object.

\begin{xca}[Pasting lemma]
  Let $\mathbf{C}$ be a category.
  Consider a commutative diagram as follows:
  \begin{center}
    \begin{tikzcd}
      A \arrow{d} \arrow{r} & B \arrow[phantom, "\lrcorner"]{dr} \arrow{d} \arrow{r} & C \arrow{d} \\
      D \arrow{r} & E \arrow{r} & F
    \end{tikzcd}
  \end{center}
  Then the left square is cartesian if and only if the outer rectangle is cartesian.
  As a corollary we have the formula
  \[ (X \times_{Y} Z) \times_{Z} W \cong X \times_{Y} W. \]
  The similarity with the formula
  \[ (R \otimes_{S} S') \otimes_{S'} T \cong R \otimes_{S} T \]
  is no coincidence, as hinted at by one of the previous examples (using the equivalence between $\mathbf{Aff}^{\mathrm{op}}$ and $\mathbf{Ring}$).
\end{xca}

Moving on to some more terminology.
There are many situations in which we are working over a given object as our base.
For example, vector bundles over a topological space, schemes over another scheme, covering spaces over a topological space...
If we are working over an object $A$ in $\mathbf{C}$ and we have a morphism $f \colon B \to A$ in $\mathbf{C}$, we are sometimes interested in using this morphism as a \textit{base change} morphism and start working over $B$ instead.
In this situation we can say that $q_{1} \colon D \to B$ is the \textit{pull back} of $g \colon C \to A$ under $f$.

Let $\mathbf{P}$ be a property of morphisms in a category $\mathbf{C}$.
We say that the property $\mathbf{P}$ is \textit{stable under pull back} if the following holds:~for all cartesian squares
\begin{center}
  \begin{tikzcd}
    D \arrow[phantom, "\lrcorner"]{dr} \arrow[swap]{d}{q_{1}} \arrow{r}{q_{2}} & C \arrow{d}{g} \\
    B \arrow[swap]{r}{f} & A
  \end{tikzcd}
\end{center}
in $\mathbf{C}$, $g$ has $\mathbf{P}$ implies that $q_{1}$ has $\mathbf{P}$.
In the last talk, Tanuj sketched the proof that ``being an isomorphism'' is stable under pull back.
The same is true for monomorphisms, but not necessarily true for epimorphisms.

\begin{rem}
  A comprehensive list of properties of morphisms which are stable under pull back in algebraic geometry can be found in \cite[Appendix C]{gw10}.
\end{rem}

Let us finally come to particular examples of fiber products.
We want to abstract the theory of sheaves on topological spaces into a theory of sheaves on categories, so let us focus on the category $\mathbf{Top}$ for the remaining of this subsection.

Let $f \colon Y \to X$ and $g \colon Z \to X$ be two continuous maps between topological spaces.
Then their fiber product can be constructed as
\[ Y \times_{X} Z = \{ (y,z) \in Y \times Z \mid f(y) = g(z) \}. \]
This set is endowed with the subspace topology and the projections $p_{1} \colon Y \times_{X} Z \to Y$ and $p_{2} \colon Y \times_{X} Z \to Z$ are the restrictions of the projections from the cartesian product.

\begin{exmp}\label{exmp:preimage}
  Let $X$ be a topological space and let $i \colon U \to X$ be the inclusion of an open subset.
  Let $f \colon Y \to X$ be a continuous map.
  Then:
  \begin{center}
    \begin{tikzcd}
      f^{-1}(U) \arrow[phantom, "\lrcorner"]{dr} \arrow[hookrightarrow]{r} \arrow[swap]{d}{f|_{f^{-1}(U)}} & Y \arrow{d}{f} \\
      U \arrow[hookrightarrow]{r} & X
    \end{tikzcd}
  \end{center}
  Note that $f^{-1}(U)$ is not equal to the fiber product as described above, cf.~\Cref{rem:cartesian}.
\end{exmp}

A particular example of the previous example is the following:

\begin{exmp}
  Let $X$ be a topological space and let $i \colon U \to X$ and $j \colon V \to X$ be inclusions of two open subsets.
  Then:
  \begin{center}
    \begin{tikzcd}
      U \cap V \arrow[phantom, "\lrcorner"]{dr} \arrow[hookrightarrow]{r} \arrow[hookrightarrow]{d} & V \arrow[hookrightarrow]{d} \\
      U \arrow[hookrightarrow]{r} & X
    \end{tikzcd}
  \end{center}
  Again, cf.~\Cref{rem:cartesian}.
\end{exmp}

\begin{exmp}\label{exmp:fiber}
  Let $X$ be a topological space and let $i \colon \{ * \} \to X$ be a map, which corresponds to picking a point $i(*) \in X$.
  Let $f \colon Y \to X$ be a continuous map.
  Then:
  \begin{center}
    \begin{tikzcd}
      f^{-1}(i(*)) \arrow[phantom, "\lrcorner"]{dr} \arrow{r} \arrow[hookrightarrow]{d} & \{ * \} \arrow{d}{i} \\
      Y \arrow[swap]{r}{f} & X
    \end{tikzcd}
  \end{center}
\end{exmp}

\begin{exmp}
  Let $X$ be a topological space and let $p \colon E \to X$ be a vector bundle on $X$.
  Let $f \colon Y \to X$ be a continuous map.
  Then we define the pull back $f^{*}E \to Y$ of $E$ along $f$ as
  \begin{center}
    \begin{tikzcd}
      f^{*}E \arrow[phantom, "\lrcorner"]{dr} \arrow{d} \arrow{r} & E \arrow{d}{p} \\
      Y \arrow[swap]{r}{f} & X
    \end{tikzcd}
  \end{center}
  The morphism $f^{*}E \to Y$ is a vector bundle on $Y$ and the fiber of $f^{*}E \to Y$ over a point $y \in Y$ is isomorphic (I really want to say ``the same'' here) as the fiber of $E \to X$ over $f(y)$.
\end{exmp}

\begin{exmp}
  Let $X$ be a connected topological space and let $p \colon \tilde{X} \to X$ be a covering space.
  Let's say it is a 7-sheeted covering space, i.e., each fiber has exactly 7 points.
  Then $p^{*}\tilde{X} \to \tilde{X}$ is again a 7-sheeted covering of $\tilde{X}$, so the composition $p^{*}\tilde{X} \to X$ is a 49-sheeted covering of $X$.
\end{exmp}

Related to the previous two examples, we also have:

\begin{xca}
  The homotopy lifting property is stable under pull back in $\mathbf{Top}$, i.e., the pull back of a Hurewicz fibration is again a Hurewicz fibration.
\end{xca}

To close this subsection, let us mention a couple more facts that we may or may not need to use later on:

\begin{exmp}
  Let $\mathbf{C}$ be a category with a terminal object $T$ and let $A$ and $B$ be objects in $\mathbf{C}$.
  Then:
  \begin{center}
    \begin{tikzcd}
      A \times B \arrow[phantom, "\lrcorner"]{dr} \arrow{r} \arrow{d} & B \arrow{d} \\
      A \arrow{r} & T
    \end{tikzcd}
  \end{center}
  This applies for instance to $\mathbf{C} = \mathbf{Top}$ with $T = \{ * \}$.
\end{exmp}

\begin{xca}
  Construct the fiber product of a diagram
  \begin{center}
    \begin{tikzcd}
      & C \arrow{d} \\
      B \arrow{r} & A
    \end{tikzcd}
  \end{center}
  of sets using only products and equalizers.
  (There is nothing special about the category of sets nor about the fiber product; we've already seen in previous talks that this can be done for every kind of limit in any category.
  But the computations are simpler in this case.)
\end{xca}

\subsection{The Yoneda embedding}

Let $\mathbf{C}$ be a category and let $X$ be an object in $\mathbf{C}$.
We denote by $h_{X}$ the presheaf
\[ Y \mapsto h_{X}(Y) := \operatorname{Hom}(Y,X) \]
with restriction morphisms $h_{X}(f) = (-) \circ f$.
If $f \colon X \to Y$ is a morphism in $\mathbf{C}$, then we obtain a morphism $h_{f} \colon h_{X} \to h_{Y}$ given by $f \circ (-)$.
In this way we obtain a functor
\[ h \colon \mathbf{C} \to \mathbf{PSh}(\mathbf{C}), \]
and as a consequence of the Yoneda lemma we know that this is a fully faithful functor.
This means that we may as well think of the object $X$ as the presheaf $h_{X}$, and we do not lose any information by doing so.

Suppose that we have now a diagram
\begin{center}
  \begin{tikzcd}
    & Z \arrow{d} \\
    Y \arrow{r} & X
  \end{tikzcd}
\end{center}
in $\mathbf{C}$.
It may well happen that the fiber product $Y \times_{X} Z$ does not exist in $\mathbf{C}$.
But we have seen in previous talks that limits of presheaves always exist and can be computed point-wise, so the fiber product $h_{Y} \times_{h_{X}} h_{Z}$ does exist in $\mathbf{PSh}(\mathbf{C})$.
Explicitly, this is the presheaf given on objects by
\[ T \mapsto \operatorname{Hom}(T,Y) \times_{\operatorname{Hom}(T,X)} \operatorname{Hom}(T,Z). \]
The fiber product $Y \times_{X} Z$ exists in $\mathbf{C}$ if and only if this presheaf is representable, i.e., if there exists an object $L$ in $\mathbf{C}$ such that $h_{L} \cong h_{Y} \times_{h_{X}} h_{Z}$.
Spelling out the definitions we see that in this case such an object $L$ is necessarily a fiber product $Y \times_{X} Z$.

\begin{cav}
  From now on we may allow ourselves a slight abuse of terminology and notation and think and talk about an object $X$ in a category $\mathbf{C}$ as if it was really the same as the corresponding presheaf $h_{X}$.
  So we may say stuff like ``the fiber product $Y \times_{X} Z$ is representable'' instead of saying ``the fiber product $Y \times_{X} Z$ exists''.
  Forcing ourselves to have this flexibility may make some of the upcoming concepts more natural.
\end{cav}

\subsection{Pretopologies}

Let $X$ be a topological space and let $\mathscr{F}$ be a presheaf on $X$.
Let $U$ be an open subset of $X$ and let $\{U_{i}\}_{i \in I}$ be an open covering of $U$.
First define a morphism
\begin{align*}
  \rho \colon \mathscr{F}(U) & \to \prod_{i \in I} \mathscr{F}(U_{i}) \\
  s & \mapsto (s|_{U_{i}})_{i \in I}
\end{align*}
induced by each restriction morphism $\rho^{U}_{U_{i}} \colon \mathscr{F}(U) \to \mathscr{F}(U_{i}), s \mapsto s|_{U_{i}}$ and the universal property of the product.
Next we want to restrict each family of sections $(s_{i})_{i \in I}$ in $\prod_{i \in I} \mathscr{F}(U_{i})$ to the intersections $U_{i} \cap U_{j}$.
There are two ways to do this.
The first one corresponds to the morphism
\begin{align*}
  \sigma_{1} \colon \prod_{i \in I} \mathscr{F}(U_{i}) & \to \prod_{(i, j) \in I \times I} \mathscr{F}(U_{i} \cap U_{j}) \\
  (s_{i})_{i \in I} & \mapsto \left(\rho^{U_{i}}_{U_{i} \cap U_{j}}(s_{i})\right)_{(i, j) \in I \times I},
\end{align*}
and the second one corresponds to the morphism
\begin{align*}
  \sigma_{2} \colon \prod_{i \in I} \mathscr{F}(U_{i}) & \to \prod_{(i, j) \in I \times I} \mathscr{F}(U_{i} \cap U_{j}) \\
  (s_{i})_{i \in I} & \mapsto \left(\rho^{U_{j}}_{U_{i} \cap U_{j}}(s_{j})\right)_{(i, j) \in I \times I}.
\end{align*}
The presheaf $\mathscr{F}$ is a sheaf if and only if for every $U$ and every open cover $\{U_{i}\}_{i \in I}$ the diagram
\begin{center}
  \begin{tikzcd}
    \mathscr{F}(U) \arrow{r}{\rho} & \prod_{i \in I} \mathscr{F}(U_{i}) \arrow[shift left=2]{r}{\sigma_{1}} \arrow[swap, shift right=2]{r}{\sigma_{2}} & \prod_{(i, j) \in I \times I} \mathscr{F}(U_{i} \cap U_{j})
  \end{tikzcd}
\end{center}
is exact, meaning that the left arrow is an equalizer of the two parallel arrows on the right.
We will refer to this as the \textit{sheaf condition} with respect to the open cover $\{ U_{i} \}_{i \in I}$.

In order to state the sheaf condition we had to use the notion of an open cover of an open subset in $X$ and we needed to consider intersections of elements in the open cover as well.
So we were looking at open covers of objects in $\mathbf{Op}(X)$ and we were considering fiber products of elements of the cover.
We axiomatize the relevant properties of open covers in order to make sense of the sheaf condition in arbitrary categories other than $\mathbf{Op}(X)$ for a topological space $X$.

In particular, we will need to assume the existence of fiber products.
So it is convenient to introduce the following definition first:

\begin{defn}[Quarrable morphism]
  Let $\mathbf{C}$ be a category.
  A morphism $f \colon X \to Y$ in $\mathbf{C}$ is called \textit{quarrable} if for every morphism $Z \to Y$ in $\mathbf{C}$ the fiber product $X \times_{Y} Z$ is representable.
\end{defn}

We are now in a more comfortable position to introduce the following:

\begin{defn}[Pretopology]
  Let $\mathbf{C}$ be a category.
  A \textit{pretopology} on $\mathbf{C}$ consists of the data, for each object $X$ in $\mathbf{C}$, of a set $\operatorname{Cov}(X)$ of sets $\{ X_{i} \to X \}_{i \in I}$ of morphism in $\mathbf{C}$ with target $X$, subject to the following axioms:
  \begin{enumerate}[label=PT \arabic*)]\addtocounter{enumi}{-1}
    \item For every object $X$ in $\mathbf{C}$, every $\{ X_{i} \to X \}_{i \in I}$ in $\operatorname{Cov}(X)$ and every $i \in I$, the morphism $X_{i} \to X$ is quarrable.\label{pt:0}
    \item For every object $X$ in $\mathbf{C}$, every $\{ X_{i} \to X \}_{i \in I}$ in $\operatorname{Cov}(X)$ and every morphism $f \colon Y \to X$ in $\mathbf{C}$, the set $\{ X_{i} \times_{X} Y \to Y \}_{i \in I}$ is in $\operatorname{Cov}(Y)$.
      (Stability under base change.)\label{pt:1}
    \item Let $X$ be an object in $\mathbf{C}$, let $\{ X_{i} \to X \}_{i \in I}$ be a set in $\operatorname{Cov}(X)$ and for each $i \in I$ let $\{ X_{i,j} \to X_{i}\}_{j \in J_{i}}$ be a set in $\operatorname{Cov}(X_{i})$.
      Then the set of compositions $\{ X_{i,j} \to X \}_{i \in I, j \in J_{i}}$ is in $\operatorname{Cov}(X)$.
      (Stability under composition.)\label{pt:2}
    \item The collection $\{ X \xrightarrow{\operatorname{id}_{X}} X \}$ is in $\operatorname{Cov}(X)$.\label{pt:3}
  \end{enumerate}
\end{defn}

We will call a collection $\{ X_{i} \to X \}_{i \in I}$ in $\operatorname{Cov}(X)$ a \textit{covering} of $X$ in $\mathbf{C}$ (endowed with this pretopology).
We will also use $\operatorname{Cov}(\mathbf{C})$ to denote the collection of all coverings in a given pretopology on $\mathbf{C}$.

\begin{rem}
  This is the definition in \cite{sga4}.
  In \cite{stacks-project}, condition \ref{pt:3} is apparently strengthened a bit to ask that every isomorphism $\{ Y \xrightarrow{\cong} X \}$ is in $\operatorname{Cov}(X)$.
  But this follows from \ref{pt:3}, \ref{pt:1} and the cartesian square
  \begin{center}
    \begin{tikzcd}
      Y \arrow[phantom, "\lrcorner"]{dr} \arrow{r} \arrow{d} & X \arrow[equal]{d} \\
      X \arrow[equal]{r} & X
    \end{tikzcd}
  \end{center}
\end{rem}

\begin{xca}
  Convince yourself that open coverings of an open subset of a topological space have the properties above.
\end{xca}

Once we have endowed our category $\mathbf{C}$ with a pretopology, we can ask ourselves whether a given presheaf $\mathscr{F}$ in $\mathbf{PSh}(\mathbf{C})$ is a \textit{sheaf} with respect to this pretopology.
The definition is the same as it was for topological spaces.
For each object $X$ in $\mathbf{C}$ and each covering $\{ X_{i} \xrightarrow{f_{i}} X \}_{i \in I}$ we define the morphism $\rho \colon \mathscr{F}(X) \to \prod_{i \in I} \mathscr{F}(X_{i})$ by applying the universal property of the product to the morphisms $\mathscr{F}(f_{i}) \colon \mathscr{F}(X) \to \mathscr{F}(X_{i})$.
For each $(i, j) \in I \times I$ we have a cartesian diagram as follows:
\begin{center}
  \begin{tikzcd}
    X_{i} \times_{X} X_{j} \arrow[phantom, "\lrcorner"]{dr} \arrow{r}{p_{2}^{(i,j)}} \arrow[swap]{d}{p_{1}^{(i,j)}} & X_{j} \arrow{d}{f_{j}} \\
    X_{i} \arrow[swap]{r}{f_{i}} & X
  \end{tikzcd}
\end{center}
The first morphism $\sigma_{1} \colon \prod_{i \in I} \mathscr{F}(X_{i}) \to \prod_{(i, j) \in I \times I} \mathscr{F}(X_{i} \times_{X} X_{j})$ is the one induced by the morphisms $\mathscr{F}(p_{1}^{(i,j)})$, and the second morphism $\sigma_{2}$ is the one induced by the morphisms $\mathscr{F}(p_{2}^{(i,j)})$.
The presheaf $\mathscr{F}$ is then a sheaf if and only if for every object $X$ in $\mathbf{C}$ and every cover $\{ X_{i} \to X \}_{i \in I}$ in $\operatorname{Cov}(X)$ the diagram
\begin{center}
  \begin{tikzcd}
    \mathscr{F}(X) \arrow{r}{\rho} & \prod_{i \in I} \mathscr{F}(X_{i}) \arrow[shift left=2]{r}{\sigma_{1}} \arrow[swap, shift right=2]{r}{\sigma_{2}} & \prod_{(i, j) \in I \times I} \mathscr{F}(X_{i} \times_{X} X_{j})
  \end{tikzcd}
\end{center}
is exact.
We will refer to this as the \textit{sheaf condition} with respect to the cover $\{ X_{i} \to X \}_{i \in I}$.

\begin{lm}\label{lm:refinement}
  Let $X$ be a topological space and let $\mathscr{F}$ be a presheaf on $X$.
  Let $U \subseteq X$ be an open subset and let $\{ U_{i} \}_{i \in I}$ be an open cover of $U$.
  Let $\{ V_{j} \}_{j \in J}$ be a refinement of $\{ U_{i} \}_{i \in I}$, i.e., $\{ V_{j} \}_{j \in J}$ is still an open cover of $U$ and for every $j \in J$ there exists some $i \in I$ such that $V_{j} \subseteq U_{i}$.
  \begin{enumerate}[label=(\alph*)]
    \item If $\mathscr{F}(U) \to \prod_{j \in J} \mathscr{F}(V_{j})$ is injective, then so is $\mathscr{F}(U) \to \prod_{i \in I} \mathscr{F}(U_{i})$.
    \item If $\mathscr{F}$ satisfies the sheaf condition with respect to $\{ V_{j} \}_{j \in J}$ and for every $i \in I$ the map $\mathscr{F}(U_{i}) \to \prod_{j \in J} \mathscr{F}(V_{j} \cap U_{i})$ is injective, then $\mathscr{F}$ satisfies the sheaf condition with respect to $\{ U_{i} \}_{i \in I}$.
  \end{enumerate}
  \begin{proof}
    Using the axiom of choice we may fix a function $\beta \colon J \to I$ such that $V_{j} \subseteq U_{\beta(j)}$ for all $j \in J$.

    We first show $(a)$.
    Suppose that $s, t \in \mathscr{F}(U)$ are sections such that $s|_{U_{i}} = t|_{U_{i}}$ for all $i \in I$.
    Then we have
    \[ s|_{V_{j}} = (s|_{U_{\beta(j)}})|_{V_{j}} = (t|_{U_{\beta(j)}})|_{V_{j}} = t|_{V_{j}} \]
    for all $j \in J$, so $s = t$.

    Let us show $(b)$ now.
    The sheaf condition for $\{ V_{j} \}_{j \in J}$ tells us that for every family of sections $(s_{j})_{j \in J} \in \prod_{j \in J} \mathscr{F}(V_{j})$ such that for each $(j_{1},j_{2}) \in J \times J$ we have
    \[ s_{j_{1}}|_{V_{j_{1}} \cap V_{j_{2}}} = s_{j_{2}}|_{V_{j_{1}} \cap V_{j_{2}}} \]
    there exists a unique section $s \in \mathscr{F}(U)$ such that $s_{j} = S|_{V_{j}}$ for all $j \in J$.
    To check the sheaf condition for $\{ U_{i} \}_{i \in I}$, let now $(s_{i})_{i \in I} \in \prod_{i \in I}\mathscr{F}(U_{i})$ be a family of sections such that for every $(i_{1}, i_{2}) \in I \times I$ we have
    \[ s_{i_{1}}|_{U_{i_{1}} \cap U_{i_{2}}} = s_{i_{2}}|_{U_{i_{1}} \cap U_{i_{2}}}. \]
    We define a collection $(t_{j})_{j \in J} \in \prod_{j \in J} \mathscr{F}(V_{j})$ as follows.
    Then we set $t_{j} := s_{\beta(j)}|_{V_{j}}$ for each $j \in J$.
    So we have a collection $(t_{j})_{j \in J} \in \prod_{j \in J} \mathscr{F}(V_{j})$.
    Then we have
    \[ t_{j_{1}}|_{V_{j_{1}} \cap V_{j_{2}}} = (s_{\beta(j_{1})}|_{U_{\beta(j_{1})} \cap U_{\beta(j_{2})}})|_{V_{j_{1}} \cap V_{j_{2}}} = (s_{\beta(j_{2})}|_{U_{\beta(j_{1})} \cap U_{\beta(j_{2})}})|_{V_{j_{1}} \cap V_{j_{2}}} = t_{j_{2}}|_{V_{j_{1}} \cap V_{j_{2}}} \]
    for all $(j_{1}, j_{2}) \in J \times J$.
    Hence there exists a unique $s \in \mathscr{F}(U)$ such that $t_{j} = s|_{V_{j}}$ for all $j \in J$.
    It remains to show now that $s|_{U_{i}} = s_{i}$ for all $i \in I$.
    By assumption, the map $\mathscr{F}(U_{i}) \to \prod_{j \in J} \mathscr{F}(V_{j} \cap U_{i})$ is injective for every $i \in I$, so it suffices to show that $s|_{U_{i}}$ and $s_{i}$ have the same image under this map.
    But now
    \[ s|_{V_{j} \cap U_{i}} = t_{j}|_{V_{j} \cap U_{i}} = (s_{\beta(j)}|_{U_{\beta(j)} \cap U_{i}})|_{V_{j} \cap U_{i}} = (s_{i}|_{U_{\beta(j)} \cap U_{i}})|_{V_{j} \cap U_{i}} = s_{i}|_{V_{j} \cap U_{i}}. \]
  \end{proof}
\end{lm}

\begin{rem}
  Essentially the same proof works in the context of categories and pretopologies, requiring only some slightly more involved notation and some extra assumptions to ensure existence of fiber products, cf.~\cite[\href{https://stacks.math.columbia.edu/tag/0G1L}{Tag 0G1L}]{stacks-project}.
\end{rem}

\begin{cor}\label{cor:refinement}
  Let $\mathbf{C}$ be a category.
  For simplicity, let us assume that all fiber products exist in $\mathbf{C}$.
  Let $\operatorname{Cov}_{1}(\mathbf{C})$ and $\operatorname{Cov}_{2}(\mathbf{C})$ be two pretopologies on $\mathbf{C}$ and assume that every cover $\{ U_{i} \xrightarrow{f_{i}} X \}_{i \in I}$ in $\operatorname{Cov}_{1}(\mathbf{C})$ admits a refinement $\{ V_{j} \xrightarrow{g_{j}} X \}_{j \in J}$ in $\operatorname{Cov}_{2}(\mathbf{C})$, meaning that there exists a function $\beta \colon J \to I$ and for each $j \in J$ a morphism $\alpha_{j} \colon V_{j} \to U_{\beta(j)}$ such that the diagram
  \begin{center}
    \begin{tikzcd}
      V_{j} \arrow[swap]{dr}{g_{j}} \arrow{r}{\alpha_{j}} & U_{\beta(j)} \arrow{d}{f_{\beta(j)}} \\
      & X
    \end{tikzcd}
  \end{center}
  commutes.
  If a presheaf $\mathscr{F}$ is a sheaf with respect to the pretopology given by $\operatorname{Cov}_{2}(\mathbf{C})$, then it is also a sheaf with respect to the pretopology given by $\operatorname{Cov}_{1}(\mathbf{C})$.
  \begin{proof}
    Part $(a)$ of \Cref{lm:refinement} ensures that $\mathscr{F}$ is separated with respect to the pretopology $\operatorname{Cov}_{1}(\mathbf{C})$, which in turn allows us to apply part $(b)$ of \Cref{lm:refinement}.
  \end{proof}
\end{cor}

This already hints at the main motivation point for this talk:~different pretopologies may give rise to the same notion of sheaf on a category.
A bit like when one defines a notion of smooth function on a topological manifold; different smoothly compatible atlases may give rise to the same notion of smooth function.
A way to solve this is to define a smooth structure on a topological manifold as a maximal smoothly compatible atlas.
This solution has a similar flavour to the one that we will adopt in our situation; we will come back to this later on.

\begin{exmp}
  Let $X$ be a set and consider the \textit{indiscrete topology} $\{ \varnothing, X \}$ on $X$.
  If $U \subseteq X$ is an open subset, the only possible open covers of $U$ are $\{ U \}$ and $\{ \varnothing, U \}$.
  If we want $\mathscr{F}$ to be sheaf, we need at the very least that $\mathscr{F}(\varnothing) = \{ * \}$, because we can always take the empty covering of the empty set and an empty product of sets is a terminal object in $\mathbf{Set}$, i.e., a singleton.
  But other than that, there is no restriction on $\mathscr{F}$ to be a sheaf.
  Indeed, assume $\mathscr{F}(\varnothing) = \{ * \}$ and let $U \subseteq X$ be a non-empty open subset.
  Let us first consdier the open cover $\{ U \}$ of $U$.
  Let $\Delta_{U} = \{ (x,y) \in U \times U \mid x = y \}$, which is isomorphic to $U$.
  Then we have a cartesian square
  \begin{center}
    \begin{tikzcd}
      \Delta_{U} \arrow[phantom, "\lrcorner"]{dr} \arrow[swap]{d}{p_{1}} \arrow{r}{p_{2}} & U \arrow[equal]{d} \\
      U \arrow[equal]{r} & U
    \end{tikzcd}
  \end{center}
  The sheaf condition with respect to this cover translates into the diagram
  \begin{center}
    \begin{tikzcd}
      \mathscr{F}(U) \arrow[equal]{r} & \mathscr{F}(U) \arrow[shift left=2]{r}{p_{1}^{*}} \arrow[shift right=2, swap]{r}{p_{2}^{*}} & \mathscr{F}(\Delta_{U})
    \end{tikzcd}
  \end{center}
  being exact, where we use the shorthand notation $p_{i}^{*}$ instead of $\mathscr{F}(p_{i})$ for each $i \in \{ 1, 2 \}$.
  Commtuativity of the cartesian diagram above implies that $p_{1} = p_{2}$, hence $p_{1}^{*} = p_{2}^{*}$ and the diagram is indeed exact.
  Let us now consider the cover $\{ \varnothing, U \}$ of $U$.
  Then we want the following diagram to be exact:
  \begin{center}
    \begin{tikzcd}
      \mathscr{F}(U) \arrow{r}{\rho} & \mathscr{F}(\varnothing) \times \mathscr{F}(U) \arrow[shift left=2]{d}{\sigma_{1}} \arrow[swap, shift right=2]{d}{\sigma_{2}} \\
      & \mathscr{F}(\varnothing \times_{U} \varnothing) \times \mathscr{F}( \varnothing \times_{U} U) \times \mathscr{F}(U \times_{U} \varnothing) \times \mathscr{F}(U \times_{U} U)
    \end{tikzcd}
  \end{center}
  The morphism $\rho \colon \mathscr{F}(U) \to \{ * \} \times \mathscr{F}(U)$ is an isomorphism, so we need to check that $\sigma_{1} = \sigma_{2}$.
  Let $(*,s) \in \mathscr{F}(\varnothing) \times \mathscr{F}(U)$.
  Then we have
  \[ \sigma_{1}(*,s) = (*,*,*,p_{1}^{*}(s)) = (*,*,*,p_{2}^{*}(s)) = \sigma_{2}(*,s), \]
  hence this second diagram is exact as well.

  This whole discussion shows that the presheaf $\mathscr{F}$ is a sheaf if and only if $\mathscr{F}(\varnothing) = \{ * \}$.
  Being a sheaf in the topological sense is the same as being a sheaf in the category $\mathbf{Op}(X)$ with the pretopology $\operatorname{Cov}_{1}$ given by all possible open covers of all open subsets.
  Let now $\operatorname{Cov}_{2}$ be the \textit{indiscrete pretopology} on $\mathbf{Ob}(X)$,
  in which $\{ U \}$ is the only cover of an open subset $U \subseteq X$.
  The difference here is that we are not allowing the empty covering of the empty set.
  We are in the assumptions of \Cref{cor:refinement}, because every cover of an open subset $U \subseteq X$ in the indiscrete pretopology can be refined by an open cover, i.e., by a cover in the pretoplogy $\operatorname{Cov}_{1}$.
  So every presehaf $\mathscr{F}$ which is a sheaf in the topological sense, i.e., with respect to the pretopology $\operatorname{Cov}_{1}$, is also a sheaf with respect to the indiscrete pretopology $\operatorname{Cov}_{2}$.
  Indeed, in this case this is easily seen to be the case, because every presheaf $\mathscr{F}$ with $\mathscr{F}(\varnothing) = \{ * \}$ is in particular a presheaf.
\end{exmp}

\begin{exmp}
  Let $X$ be a topological space and let $\mathcal{B}$ be a basis for the topology, i.e., a collection of open subsets of $X$ such that
  \begin{enumerate}
    \item The elements of $\mathcal{B}$ cover $X$, i.e., for every $x \in X$ there exists some $B \in \mathcal{B}$ such that $x \in B$.
    \item Given two elements $B_{1}, B_{2} \in \mathcal{B}$ and given a point $x \in B_{1} \cap B_{2}$, there exists some $B_{3} \in \mathcal{B}$ such that $x \in B_{3}$ and $B_{3} \subseteq B_{1} \cap B_{2}$.
  \end{enumerate}
  We take $\operatorname{Cov}_{1}$ as the topological pretopology on $\mathbf{Op}(X)$, in which the coverings are just the open covers of open subsets of $X$; and we take $\operatorname{Cov}_{2}$ as the pretopology on $\mathbf{Op}(X)$ in which the open covers are the open covers of open subsets of $X$ whose elements are all in $\mathcal{B}$.
  Then a presheaf $\mathscr{F}$ is a $\operatorname{Cov}_{1}$ sheaf if and only if it is a $\operatorname{Cov}_{2}$ sheaf.
  
  Indeed, we can apply \Cref{cor:refinement} to deduce the claim.
  If $\{ U_{i} \}_{i \in I}$ is a cover of $U\subseteq X$ in $\operatorname{Cov}_{2}$, then it is also a cover of $U\subseteq X$ in $\operatorname{Cov}_{1}$, because we were assuming that the elements of our basis $\mathcal{B}$ are all open.
  This shows that a $\operatorname{Cov}_{1}$ sheaf is also a $\operatorname{Cov}_{2}$ sheaf.
  (Note that in this case it is really on the nose, since $\operatorname{Cov}_{2} \subseteq \operatorname{Cov}_{1}$, so we are just checking a smaller amount of conditions.)
  Conversely, every usual open cover of $U \subseteq X$ admits a refinement by elements of the basis $\mathcal{B}$, so every $\operatorname{Cov}_{2}$ sheaf is a $\operatorname{Cov}_{1}$ sheaf.
\end{exmp}

In fact, it would even suffice to define a presheaf on a basis for the topology to obtain a presheaf on $\mathbf{Op}(X)$, cf.~\cite[Chapter 0, (3.2.1)]{ega}.

\subsection{Sieves}




\section{Grothendieck topologies}

{\color[rgb]{0.2,0.2,0.6}
\begin{itemize}
  \item Define the notion of a \textit{Grothendieck topology} on a category and of \textit{site}.
  \item Give examples including the \textit{chaotic} topology, the \textit{discrete} topology, and the topology \textit{associated} to a given pretopology.
\end{itemize}
}

\section{Topologies and sheaves}

{\color[rgb]{0.2,0.2,0.6}
\begin{itemize}
  \item Define the topology \textit{generated} by a family of sieves.
    Dually, define the \textit{finest} topology for which all presheaves in a given family are separated/sheaves.
  \item Show that this defines an order-reversing correspondence between Grothendieck topologies and categories of sheaves.
    Deduce that a pretopology and the associated topology define the same category of sheaves.
  \item Introduce the notion of \textit{canonical} and \textit{sub-canonical} topology.
    Explain how to characterize them via the Yoneda embedding.
\end{itemize}
}

\printbibliography
\vfill

\end{document}
